\section{Experimental Details}

	\subsection{Equipment}
		The experiment apparatus consisted of a mercury filled tube situated inside an electric oven, a thermocouple with its probe positioned just above the filled tube a power supply and an analog oscilloscope.
		\\
		\\
		The main apparatus had an electric oven to increase the collision probabilities of mercury atoms and incident electrons, a thermocouple connected to a digital thermometer to measure the temperature and a cathode ray tube filled with mercury to observe the actual phenomena. The power supply was specifically tailored for this experiment and allowed us to control the filament voltage, collector voltage and the accelerating voltage. The oscilloscope allowed us to get readouts from the apparatus, generating the same plot from Figure \ref{fig: fh}. 
	
	\subsection{Procedure}
	
		The procedure consisted of turning the thermostat dial to approximately 160 degrees Celsius (the temperature values were quite unstable, we tried not to exceed 200 degrees Celsius), setting the toggle switch to 'Ramp', setting filament to approximately 6 Volts, setting the collector voltage to approximately 2 Volts, adjusting the amplification gain, adjusting the oscilloscope variables to see clear and many peaks and writing down the data measured from the maximum and minimum values of the peaks present on the oscilloscope screen. 
