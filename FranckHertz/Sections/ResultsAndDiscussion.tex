\section{Results And Discussion}

	\subsection{Discussion Of Results}
		It is clear from the data and analysis of data that our results agree with the quantum theory of atoms and proves the existence of discrete energy levels. However, there are still two important points to discuss while analyzing the data.
		\\
		\\
		The first one is the fact that we only observe the excitation process to a single energy level. When the electrons coming from the cathode have kinetic energies that are larger than 4.9 eV, excitation occurs and it is clear that each peak happens at multiples of 4.9 eV. This means that we only observe the transition to the first excited state. 
		\\
		\\
		The second point to discuss is that even though we say that we only observe the transition to the first excited state, there are multiple peaks in the data. This is because of the fact that if the incident electrons have energies that are multiples of 4.9 eV, they can excite more than one mercury atom by themselves. For example an electron with kinetic energy more than 9.8 eV can excite two mercury atoms, giving 4.9 eV of energy to each one, thus exciting them both to the first excited state.  
	
	\subsection{Discussion Of Errors}
		There are two main sources of error in this experiment. The first source of error is the incredibly unstable thermostat, which can't maintain its target temperature and constantly increases its temperature, forcing us to interfere with the thermostat to maintain our target temperature, resulting in thermal fluctuations. The second source of error is our instrument of  measurement. We have measured the maximum and minimum values of peaks present on the oscilloscope screen using our eyes which probably contributed to the error a lot.
		\\
		\\
		Using a more modern experimental apparatus and measurement device can mitigate a lot of error from data. Just using a digital oscilloscope instead of an analog one can make great difference in terms of result. 