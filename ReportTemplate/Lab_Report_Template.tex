\documentclass[12pt]{extarticle}
\usepackage[utf8]{inputenc}
\usepackage{cite}
\usepackage{blindtext}
\usepackage{graphicx} % Package to manage images
\usepackage{wrapfig} % Package to wrap text around figures
\usepackage{physics} % For the bra-ket notation.
\usepackage{amsmath} % For centering equations.
\usepackage{hyperref}
\usepackage{tikz}
\usetikzlibrary{quantikz}
\usepackage{pgfplots}
\usepackage{fancyhdr}
\usepackage{lastpage}

\pagestyle{fancy}
\fancyhf{}

%\rfoot{Page \thepage \hspace{1pt} of \pageref{LastPage}}

\hypersetup{
	colorlinks=true,
	linkcolor=blue,
	filecolor=magenta,      
	urlcolor=cyan,
	pdftitle={Name Of Experiment - Author Name},
	pdfpagemode=FullScreen,
}

\makeatletter

\renewcommand\paragraph{\@startsection{paragraph}{4}{\z@}%
	{-2.5ex\@plus -1ex \@minus -.25ex}%
	{1.25ex \@plus .25ex}%
	{\normalfont\normalsize\bfseries}}

\makeatother

\setcounter{secnumdepth}{4} % how many sectioning levels to assign numbers to

\setcounter{tocdepth}{4}    % how many sectioning levels to show in ToC


\fancyfoot[LE,RO]{\thepage}

\chead{\includegraphics[width=5cm, height=1cm]{./images/metu_logo.jpg}}



\begin{document}
	
	\clearpage
	%% temporary titles
	% command to provide stretchy vertical space in proportion
	\newcommand\nbvspace[1][3]{\vspace*{\stretch{#1}}}
	% allow some slack to avoid under/overfull boxes
	\newcommand\nbstretchyspace{\spaceskip0.5em plus 0.25em minus 0.25em}
	% To improve spacing on titlepages
	\newcommand{\nbtitlestretch}{\spaceskip0.6em}
	%\pagestyle{empty}
	\begin{center}
		\bfseries
		\nbvspace[1]
		\Huge
		{\nbtitlestretch\huge
			Name Of The Experiment}
		
		\nbvspace[1]
		\normalsize
		
		\footnotesize  
		Author: 
		\\
		Lab Partner:
		\\
		Experiment Date:
		\\
		Submission Date:
		
		
		
		
		\nbvspace[2]
		
		%\includegraphics[width=3in]{./images/image_related_to_the_experiment.jpeg}
		\nbvspace[3]
		\normalsize
		
		
		\nbvspace[1]
	\end{center}
	
	\newpage
	\tableofcontents
	\newpage
			
	\section{Introduction}
	
	Brief description of the goal, theory, and underlying mechanisms of the experiment. Introduction part must include:
	
	 •	Underlying theory and background information of the experiment
	 
	 •	Important concepts/terms/definitions used in the experiment
	 
	
	\section{Experimental Details}
	
	Brief description of the experimental setup and data acquisition. Experimental Details part must include:
	
	•	Important steps of the experimental procedure
	•	Working principles and details of the critical equipment used in the experiment
	
	
	\section{Measurement And Data Analysis}
	
	Analysis, calculations, and interpretation of the experimental data. Measurement and Data Analysis part must include:
	
	•	Data tables used during the experiment (if applicable)
	
	•	Graphs plotted to analyze the data (if applicable)
	
	•	Details of the calculations (if applicable) 
	
	
	\section{Results And Discussion}
	
	Discussion of your results and their connection to the theoretical predictions. Discussion part must include:
	
	•	A brief summary of the important steps of the experiment
	
	•	Discussion of the results and methods
	
	•	Possible reasons for errors and suggestions to avoid/reduce error in the experiment 
	
	
	\nocite{*}
	\bibliographystyle{plain} % We choose the "plain" reference style
	\bibliography{ref} % Entries are in the refs.bib file
			
		
\end{document}
