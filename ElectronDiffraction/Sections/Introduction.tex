\section{Introduction}

	\subsection{Goal}
		The purpose of this experiment is to observe how matter, besides light, behaves like a wave under some conditions as proposed by Louis de Broglie in 1924 (which pioneered the quantum mechanics formulation besides Bohr's early quantum theory) and to make predictions about the lattice spacing of the diffracting material using the data from the experiment solidifying the fact that matter can be interpreted as a wave.  
	\subsection{Theory}
		The theory behind the Electron Diffraction experiment can be investigated as two different sections, Matter Waves and Diffraction Mechanism.
		\subsubsection{Matter Waves}
			De Broglie stated that both massless and massive particles needed to satisfy a common set of relations that tied the energy, frequency, momentum and wavelength together. 
			\\
			\\
			Following from de Broglie's hypothesis, any particle with energy and momentum is also a de Broglie wave that can be written as:
			\begin{align*}
				E = h f && \lambda = \frac{h}{p}
			\end{align*}
			Expressing these relations using the wave vector $\vec{k}$, we get:
			\begin{align*}
				E = \hbar \omega && \vec{p} = \hbar \vec{k}
			\end{align*}
			Recalling from wave theory that energy is carried by the group velocity, we can say that the analogue of this is the velocity of the particle in the case of matter waves. Using the relativistic definitions of energy and momentum, it follows from de Broglie relations stated above that:
			\begin{align*}
				\lambda f = \frac{\omega}{k} = \frac{E / \hbar}{p / \hbar} = \frac{E}{p} = \frac{m c^2}{m u} = \frac{c^2}{u}
			\end{align*}
		Which satisfies that $\lambda f = c$ if the particle is massless $u = c$. 
		\subsubsection{Diffraction Mechanism}
			If electrons behave like a wave in accordance with de Broglie's hypothesis, we can make an analogy with the diffraction of x-rays by a crystal to understand the mechanics behind the electron diffraction experiment.
			\\
			\\
			Crystals have the property to act as three-dimensional gratings since they scatter waves and in return produce interference effects. An illustration of this phenomenon can be seen below.
			\\
			\\
			\begin{figure}[H]
				\includegraphics[width=\textwidth]{./images/bragg_diffraction.png}
				\caption{Illustration Of Electron Waves Reflecting From Atomic Planes}
				\label{fig:bragg}
			\end{figure}
			Following from Figure \ref{fig:bragg}, Bragg's Law can be stated as 
			\begin{align*}
				n \lambda = 2 d sin(\theta)
			\end{align*}
			where $n$ is the order of diffraction, $d$ is the interplanar distance, $\theta$ is the diffraction angle and $\lambda$ is the wavelength. 
			\\
			\\
			It is clear that Bragg's Law allows us to determine the wavelength of the waves if we know the crystal structure or the interatomic spacing of the crystal structure if the wavelength of the waves are known.
			\\
			\\
			Each set of parallel lattice planes present inside the crystal produces a pair of spots in the electron diffraction pattern, with the direct beam located in the center of the pattern.
		