\section{Results And Discussion}

	\subsection{Discussion Of Results}
		It is clear from the percentage error results that we have successfully identified the wavelengths of the spectral lines of Na atom in the first part. This result shows that sodium has a discrete spectra, i.e. only certain wavelengths/colors can be observed from its diffracted light.
		\\
		\\
		In the second part, we have managed to prove that fine structure exists and it is possible to observe the splitting of spectral lines if we have enough resolving power. We have managed to observe the splitting to the number order of $n = 3$ with our experimental apparatus.
		\\
		\\
		The reason for the difference of atomic spectra for different elements is the fact that they have different numbers of protons, electrons and different electron arrangements. Basically, the amounts of energy resulting from absorption and emission is different for each element.
	
	\subsection{Discussion Of Errors}
	We have lined up a very slim crosshair with the spectral lines and then measured the angle values manually on a Vernier scale mounted on the experiment apparatus. This means that we have used our eyes two times, to line up the crosshair with the spectral lines and to measure the angle values which should produce a lot of error in return. The errors can be significantly lowered just by using a digital scale instead of a manual Vernier scale which is hard to line up and relatively primitive in this day and age.
	\\
	\\
	Contrary to our suspicions though, the actual error values turned out to be shockingly small when we processed the data and calculated the percentage error. The calculated percentage error values can be seen in Table \ref{table:Data1} and Table \ref{table:Data2}.