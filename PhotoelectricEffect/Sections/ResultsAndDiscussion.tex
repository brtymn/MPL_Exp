\section{Results And Discussion}

	\subsection{Results}
	Equation of the line in Figure \ref{fig:plt2} is $f(x) = 0.005695 x - 3.911$, which in physical context means that we have $V_s(\nu) = 0.005695 \nu - 3.911$. Multiplying this equation with electron charge $e$, we know that the resulting term needs to be equal to $KE_{max} = h \nu - \Phi$. 
	\begin{equation}
		V_s(\nu) e = KE_{max} \rightarrow 0.005695 \nu e - 3.911 e = h \nu - \Phi
	\end{equation}
	Then we obtain Planck's constant and the work function as:
	\begin{equation}
		h = 0.005695e \quad \text{ and } \quad \Phi = 3.911e
	\end{equation}
	\\
	\\
	In terms of errors, the phototube was isolated quite well from external light sources but it was quite apparent that there was some charge buildup inside the photoelectric effect experiment apparatus which forced us to re-calibrate it several times during the experiment to get consistent measurements.
	
	\subsection{Discussion}
	The validity of Einstein's theory was easily recognizable even without carrying out any calculations. Just looking at raw data was enough to say that photon energy depended on the wavelength of the incident photons but plotting the data and carrying out the calculations to validate the results with already established constants solidified this claim.