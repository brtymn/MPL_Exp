\section{Experimental Details}

	\subsection{Equipment}
		
		\subsubsection{Mercury Lamp}
			A mercury arc lamp uses electric discharge through vaporized mercury to produce light. While being more energy efficient than incandescent and most fluorescent lights, mercury lamps are also be 10 to 100 times brighter than incandescent lampsand can provide intense illumination over a selected wavelength through the use of optical filters.
			\\
			\\
			Mercury arc lamps produce very high luminance and radiance output levels compared to other continuously operating light sources used in optical microscopy.
		
		\subsubsection{Optical Elements}
			\paragraph{Aperture}
				An aperture is a small opening that allows only a small portion of incident light to pass on through the other optical elements in the setup. 
				\\
				\\
				An aperture with 4mm diameter was used throughout this experiment, which can be in Figure (INSERT DIG NUM HERE.	)
			\paragraph{Optical Filter}
				An optical filter is an optical device that only allows the selected wavelengths to pass on, while blocking all the other wavelengths.
				\\
				\\
				Five different optical filters were used throughout the experiment to produce output light with different wavelengths, them being: 365nm, 405nm, 436, nm, 546nm and 577nm.
		\subsubsection{Phototube}
			A phototube is a vacuum tube that is used to sense light that are mostly replaced by semiconductor photodetectors in recent years. The phototube contains a coated cathode and a circular anode inside which are crucial for the physics behind this experiment.
			\\
			\\
			The photocathode surface needs to be coated with a substance that has a low work function. The work function being uniform is an advantage as well, but a nonuniform work function can be mitigated by using apertures before the phototube, reducing the illuminated area on the photocathode. 
	
	\subsection{Procedure}
	
		\subsubsection{Calibration Procedure}
		
		\subsubsection{Experiment Procedure}
